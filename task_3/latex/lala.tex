\documentclass[a4paper,12pt]{article}
\usepackage{graphicx}
\usepackage{amsmath}
\usepackage{geometry}
\usepackage{amsmath, amssymb}
\usepackage{float}
\usepackage{caption}
\usepackage{subcaption}
\usepackage{xcolor}
\usepackage{fancyhdr}
\usepackage{datetime2}

\definecolor{darkskyblue}{rgb}{0.0, 0.5, 1.0}
\definecolor{skyblue}{RGB}{135, 206, 235}
\usepackage{wrapfig}
\usepackage{circuitikz}

\geometry{a4paper, top=0.7in, left=1in, right=1in, bottom=1in}

\begin{document}

\pagestyle{empty} % Start with empty page style

\thispagestyle{fancy} % Apply fancy style only to the first page
\fancyhf{} % Clear header and footer
\renewcommand{\headrulewidth}{0pt}

\fancyhead[L]{
	\includegraphics[width=7cm, height=1.7cm]{IMG_20250326_151233_868.png} 
}
\fancyhead[R]{
    Name: Sritama Biswas \\
    Batch: COMETFWC011 \\
    Date: 26 march 2025 
}

\vspace{10cm}
\begin{center}
    {\LARGE \textbf{\textcolor{darkskyblue}{\\ GATE QUESTION \\ ECE 2009 Q38}}}
\end{center}

\vspace{-1cm} %adjust vertical space

\section*{\textcolor{blue}{\\Question}}
Q38) Refer to the NAND and NOR latches shown in the figure. The inputs $(P_1, P_2)$ for both the latches are first made (0,1) and then, after a few seconds, made (1,1). The corresponding stable outputs $(Q_1, Q_2)$ are:

\vspace{1cm}
\noindent
\resizebox{0.48\textwidth}{!}{
\begin{circuitikz}
    \node[nand port,scale=1.2] (NAND1) at (0,2) {};
    \node[anchor=east] at (1,2) {$Q_1$};
    \node[anchor=east] at (1,0) {$Q_2$};
    \node[nand port,scale=1.2] (NAND2) at (0,0) {};
    \draw (NAND1.in 1) -- ++(-1,0) node[anchor=east] {$P_1$};
    \draw (NAND2.in 2) -- ++(-1,0) node[anchor=east] {$P_2$};
    \draw (NAND1.out) -- ++(0,-0.5) -- ($(NAND2.in 1) +(0,0.5)$) -- (NAND2.in 1);
    \draw (NAND2.out) -- ++(0,+0.5) -- ($(NAND1.in 2) +(0,-0.5)$) -- (NAND1.in 2);
\end{circuitikz}
}\hfill
\resizebox{0.48\textwidth}{!}{
\begin{circuitikz}
    \node[nor port,scale=1.2] (NOR1) at (0,2) {};
    \node[anchor=east] at (1,2) {$Q_1$};
    \node[anchor=east] at (1,0) {$Q_2$};
    \node[nor port,scale=1.2] (NOR2) at (0,0) {};
    \draw (NOR1.in 1) -- ++(-1,0) node[anchor=east] {$P_1$};
    \draw (NOR2.in 2) -- ++(-1,0) node[anchor=east] {$P_2$};
    \draw (NOR1.out) -- ++(0,-0.5) -- ($(NOR2.in 1) +(0,0.5)$) -- (NOR2.in 1);
    \draw (NOR2.out) -- ++(0,+0.5) -- ($(NOR1.in 2) +(0,-0.5)$) -- (NOR1.in 2);
\end{circuitikz}
}

\vspace{1cm}
\textbf{Options:}
\begin{enumerate}
    \item[(A)] NAND: first (0,1) then (0,1), NOR: first (1,0) then (0,0)
    \item[(B)] NAND: first (1,0) then (1,0), NOR: first (1,0) then (1,0)
    \item[(C)] NAND: first (1,0) then (1,0), NOR: first (1,0) then (0,0)
    \item[(D)] NAND: first (1,0) then (1,1), NOR: first (0,1) then (0,1)
\end{enumerate}

\section*{\textcolor{blue}{Solution}}

\subsection*{1. NAND Latch Analysis}
The NAND latch consists of two cross-coupled NAND gates, with outputs given by:

\[
Q_1 = \overline{P_1 \cdot Q_2}, \quad Q_2 = \overline{P_2 \cdot Q_1}
\]

\textbf{Step 1: Inputs (0,1)}
\[
Q_1 = \overline{0 \cdot Q_2} = \overline{0} = 1
\]
\[
Q_2 = \overline{1 \cdot Q_1} = \overline{1} = 0
\]
\textbf{Output:} \( (Q_1, Q_2) = (1,0) \)

\textbf{Step 2: Inputs (1,1)}
\[
Q_1 = \overline{1 \cdot Q_2} = \overline{1 \cdot 0} = \overline{0} = 1
\]
\[
Q_2 = \overline{1 \cdot Q_1} = \overline{1 \cdot 1} = \overline{1} = 0
\]
\textbf{Output:} \( (Q_1, Q_2) = (1,0) \)

\begin{table}[h]
    \centering
    \renewcommand{\arraystretch}{1.2}
    \begin{tabular}{|c|c|c|}
        \hline
        $P_1$ & $P_2$ & $\text{NAND Output } (Q)$ \\
        \hline
        0 & 0 & 1 \\
        0 & 1 & 1 \\
        1 & 0 & 1 \\
        1 & 1 & 0 \\
        \hline
    \end{tabular}
    \caption{NAND Gate Truth Table}
\end{table}

\subsection*{2. NOR Latch Analysis}
The NOR latch consists of two cross-coupled NOR gates, with outputs given by:

\[
Q_1 = \overline{P_1 + Q_2}, \quad Q_2 = \overline{P_2 + Q_1}
\]

\textbf{Step 1: Inputs (0,1)}
\[
Q_1 = \overline{0 + Q_2} = \overline{0} = 1
\]
\[
Q_2 = \overline{1 + Q_1} = \overline{1} = 0
\]
\textbf{Output:} \( (Q_1, Q_2) = (1,0) \)

\textbf{Step 2: Inputs (1,1)}
\[
Q_1 = \overline{1 + Q_2} = \overline{1} = 0
\]
\[
Q_2 = \overline{1 + Q_1} = \overline{1} = 0
\]
\textbf{Output:} \( (Q_1, Q_2) = (0,0) \)

\begin{table}[h]
    \centering
    \renewcommand{\arraystretch}{1.2}
    \begin{tabular}{|c|c|c|}
        \hline
        $P_1$ & $P_2$ & $\text{NOR Output } (Q)$ \\
        \hline
        0 & 0 & 1 \\
        0 & 1 & 0 \\
        1 & 0 & 0 \\
        1 & 1 & 0 \\
        \hline
    \end{tabular}
    \caption{NOR Gate Truth Table}
\end{table}

\section*{\textcolor{blue}{Final Answer}}
- \textbf{NAND latch:} First (1,0), then (1,0)
- \textbf{NOR latch:} First (1,0), then (0,0)

\textbf{\textcolor{green}{Correct Option:} (C)}

\end{document}
